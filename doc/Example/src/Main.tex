% Copyright (C) 2022 Anton Rodomanov
%
% This file is part of the `LabelNamespaces` package.
%
% You may distribute and/or modify this code under the conditions
% of the LaTeX Project Public License, version 1.3c or any later version.

\documentclass{article}

\usepackage{lipsum}
\usepackage{amsmath}
\usepackage{amsthm}
\usepackage{LabelNamespaces}
\usepackage{hyperref}
\usepackage{cleveref}

\newtheorem{theorem}{Theorem}
\newtheorem{lemma}{Lemma}
\theoremstyle{remark}
\newtheorem{remark}{Remark}

\begin{document}
  First, let us recall the following important property of real multiplication.

  \begin{lemma}\label{th:MultplicationProperty}
    For any real numbers $a, b$ and $t \geq 0$, we have
    \[
      a \leq b
      \qquad \implies \qquad
      t a \leq t b.
    \]
  \end{lemma}

  Using the above property, we can easily prove the main result.

  \begin{theorem}\label{th:MainTheorem}
    \UsingNamespace{MainTheorem}

    Let $(a_k)_{k = 0}^\infty$ be a nonnegative sequence such that
    \begin{equation}\LocalLabel{eq:RecurrenceRelation}
      a_{k + 1} \leq q a_k,
    \end{equation}
    for all $k \geq 0$, where $q \in (0, 1)$. Then, for all $k \geq 0$, we have
    \begin{equation}\LocalLabel{eq:ExplicitRate}
      a_k \leq q^k a_0.
    \end{equation}
  \end{theorem}

  \begin{proof}
    \UsingNestedNamespace{MainTheorem}{Proof}

    Let us prove \cref{\MasterName{eq:ExplicitRate}} by induction on~$k$.

    Clearly, \cref{\MasterName{eq:ExplicitRate}} is valid for $k = 0$
    since $q^0 = 1$.

    Now suppose that \cref{\MasterName{eq:ExplicitRate}} has already been proved
    for some index~$k \geq 0$. Combining
    \cref{th:MultplicationProperty,\MasterName{eq:ExplicitRate}},
    we obtain
    \begin{equation}\LocalLabel{eq:PreliminaryStep}
      q a_k \leq q (q^k a_0) = q^{k + 1} a_0.
    \end{equation}
    Putting together
    \cref{\MasterName{eq:RecurrenceRelation},\LocalName{eq:PreliminaryStep}},
    we get
    \[
      a_{k + 1} \leq q a_k \leq q^{k + 1} a_0.
    \]
    This is exactly \cref{\MasterName{eq:ExplicitRate}} for index~$k + 1$.
  \end{proof}

  \begin{remark}
    \label{th:MinorRemark}

    According to \cref{MainTheorem::eq:ExplicitRate},
    the sequence~$(a_k)_{k = 0}^{\infty}$ is bounded from above
    by a geometric progression. Also, note that, in
    \cref{MainTheorem::Proof::eq:PreliminaryStep}, we have implicitly used the
    associativity of real multiplication: $a (b c) = (a b) c$.
  \end{remark}
\end{document}